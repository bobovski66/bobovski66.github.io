
\documentclass{article}
\usepackage{hyperref}
\usepackage{amsmath,listings,mdframed}
\usepackage[margin=1in]{geometry}

\title{Proper Academic Use of Large Language Models (LLMs) and AI in Math Courses}
\author{Community College Mathematics Department}
\date{\today}

\begin{document}

\maketitle

\tableofcontents

\section{Introduction}

The rapid advancement of artificial intelligence (AI) technologies and Large Language Models (LLMs), such as ChatGPT and similar tools, has significantly impacted various fields, including education. These technologies offer powerful resources for learning, providing instant feedback, generating ideas, and assisting with problem-solving. However, their use also raises important questions about academic integrity, ethics, and effective learning strategies. 

The purpose of this document is to outline the guidelines and best practices for the appropriate use of AI and LLMs in math courses at our community college. By adhering to these guidelines, students can leverage these tools to enhance their learning experience while maintaining high standards of academic honesty and responsibility.

\subsection{Why Discuss AI and LLMs in an Academic Context?}

AI and LLMs are increasingly accessible and widely used in educational settings. These tools have the potential to:
\begin{itemize}
    \item Provide personalized learning experiences by adapting to individual student needs.
    \item Offer diverse perspectives on complex mathematical concepts, aiding in deeper understanding.
    \item Serve as virtual tutors, providing immediate responses to student queries.
\end{itemize}

However, the use of AI in academics is a double-edged sword. While these technologies can be incredibly beneficial, they also pose risks such as:
\begin{itemize}
    \item Encouraging academic dishonesty by enabling easy access to solutions without understanding.
    \item Perpetuating biases and inaccuracies present in the data used to train these models.
    \item Undermining the development of critical thinking and problem-solving skills if over-relied upon.
\end{itemize}

\subsection{Objectives of This Document}

This document aims to:
\begin{enumerate}
    \item Educate students on the ethical and effective use of AI and LLMs in their studies.
    \item Provide clear guidelines on what constitutes acceptable and unacceptable use of these technologies.
    \item Encourage a balanced approach to using AI as a tool to supplement, not replace, traditional learning methods.
    \item Foster an understanding of the ethical implications and potential biases of AI tools.
\end{enumerate}

By following the guidelines set forth in this document, students will be better prepared to use AI and LLMs responsibly and effectively, thereby enhancing their educational experience while upholding the principles of academic integrity.

\subsection{Structure of This Document}

The remainder of this document is structured as follows:
\begin{itemize}
    \item Section 2 provides an overview of what LLMs and AI are, including their capabilities and limitations.
    \item Section 3 discusses the benefits of using AI tools in education and how they can enhance learning.
    \item Section 4 outlines the proper use of LLMs and AI in academic settings, including what constitutes ethical use.
    \item Section 5 and 6 provide examples of acceptable and unacceptable uses of AI in academic work.
    \item Section 7 explains how to properly cite AI tools in academic work to avoid plagiarism.
    \item Section 8 addresses privacy and ethical considerations when using AI.
    \item Section 9 concludes with a summary of key points and encouragement to use AI responsibly.
    \item Section 10 lists additional resources for further reading on AI ethics and academic integrity.
\end{itemize}

This document is an evolving guide that reflects the dynamic nature of AI technologies and their role in education. It will be updated regularly to adapt to new developments and insights.


\section{What Are LLMs and AI?}

To understand how to use AI tools effectively in an academic setting, it is essential first to know what Large Language Models (LLMs) and artificial intelligence (AI) are, and how they function. This section provides a brief overview of these technologies and their applications in education.

\subsection{Understanding Artificial Intelligence (AI)}

Artificial Intelligence (AI) refers to the simulation of human intelligence processes by machines, particularly computer systems. These processes include learning (the acquisition of information and rules for using the information), reasoning (using rules to reach approximate or definite conclusions), and self-correction. AI has various subfields, including machine learning, where algorithms improve through experience, and natural language processing (NLP), which focuses on the interaction between computers and human languages.

\subsection{What Are Large Language Models (LLMs)?}

Large Language Models (LLMs) are a specific type of AI designed to understand, generate, and manipulate human language in a way that is contextually relevant and coherent. Examples of LLMs include GPT (Generative Pre-trained Transformer) models like ChatGPT, Claude, MetaAI, Gemini, and other similar AI tools. These models are trained on vast amounts of text data from the internet, books, articles, and more, enabling them to generate responses based on the patterns and knowledge encoded in their training data.

\subsection{How Do LLMs Work?}

LLMs work by predicting the next word in a sequence of text, given all the previous words in the sequence. This is achieved through a process called \textit{transformer architecture}, which allows the model to consider the context of each word in a sentence, thereby generating coherent and contextually appropriate responses. Key features of LLMs include:
\begin{itemize}
    \item \textbf{Text Generation:} LLMs can produce human-like text based on prompts provided by users. This includes writing essays, solving math problems, and even creating poetry or coding scripts.
    \item \textbf{Text Comprehension:} These models can understand and analyze text, summarizing content, explaining concepts, or answering questions based on provided information.
    \item \textbf{Adaptability:} LLMs can generate responses in different styles and tones, making them versatile tools for various applications.
\end{itemize}

\subsection{AI Tools for Students to Explore}

Here are some AI tools that students can try to understand their capabilities and functionalities:

\begin{itemize}
    \item \textbf{ChatGPT:} A popular LLM developed by OpenAI that can generate text, answer questions, and provide explanations on a wide range of topics. Students can try it out at \href{https://chat.openai.com/}{https://chat.openai.com/}.
    \item \textbf{Claude:} Developed by Anthropic, Claude is an AI assistant designed to be helpful, harmless, and honest. It is particularly useful for research and educational purposes. Explore Claude at \href{https://www.anthropic.com/}{https://www.anthropic.com/}.
    \item \textbf{Meta AI:} An AI research lab by Meta (formerly Facebook), which focuses on advancing the state-of-the-art in AI. Their models are often integrated into Meta’s platforms, but some are available for public use. More information can be found at \href{https://ai.facebook.com/}{https://ai.facebook.com/}.
    \item \textbf{Gemini:} Developed by Google DeepMind, Gemini is an AI that integrates various AI tools for research and problem-solving. Try it at \href{https://deepmind.com/research/gemini}{https://deepmind.com/research/gemini}.
    \item \textbf{BERT:} A model designed by Google for natural language understanding tasks. It is widely used for research and development purposes. Explore more about BERT at \href{https://ai.googleblog.com/2018/11/open-sourcing-bert-state-of-art-pre.html}{https://ai.googleblog.com/2018/11/open-sourcing-bert-state-of-art-pre.html}.
    \item \textbf{Bard AI:} Another AI tool from Google, Bard AI is designed for creative writing and storytelling, providing innovative prompts and story suggestions. Learn more at \href{https://bard.google.com/}{https://bard.google.com/}.
\end{itemize}

\subsection{Capabilities and Applications of LLMs in Education}

In an educational context, LLMs can serve multiple roles:
\begin{itemize}
    \item \textbf{Personalized Tutoring:} LLMs can provide tailored explanations and examples, helping students understand complex concepts in a way that suits their learning style.
    \item \textbf{Interactive Learning:} By engaging in dialogue with an LLM, students can explore different perspectives and receive immediate feedback, which can reinforce learning.
    \item \textbf{Resource Generation:} LLMs can generate practice questions, study guides, and summaries, supporting students in their studies.
\end{itemize}

\subsection{Limitations of LLMs and AI}

While LLMs are powerful tools, they also have several limitations:
\begin{itemize}
    \item \textbf{Lack of True Understanding:} LLMs do not understand content as humans do; they generate responses based on patterns rather than comprehension.
    \item \textbf{Potential for Misinformation:} Since LLMs are trained on a mix of reliable and unreliable sources, they may generate incorrect or misleading information.
    \item \textbf{Bias and Ethical Concerns:} The training data for LLMs may contain biases, leading to biased or inappropriate responses.
    \item \textbf{Over-Reliance Risks:} Dependence on AI for learning can undermine critical thinking and problem-solving skills if not used judiciously.
\end{itemize}

\subsection{The Role of Human Judgment in Using AI Tools}

It is crucial for students to use LLMs as a supplement to, not a replacement for, traditional learning methods. Human judgment is essential in interpreting the information provided by AI, verifying its accuracy, and applying it appropriately. Understanding these tools' capabilities and limitations enables students to harness their potential while maintaining academic integrity and rigor.

This foundational knowledge about AI and LLMs will help students make informed decisions about how to integrate these technologies into their learning process effectively.


\section{Benefits of Using AI in Education}

Artificial Intelligence (AI) and Large Language Models (LLMs) have the potential to revolutionize education by offering personalized learning experiences, enhancing student engagement, and providing new ways to understand complex concepts. This section explores the key benefits of integrating AI tools into educational settings, particularly in mathematics courses.

\subsection{Personalized Learning Experiences}

One of the most significant advantages of using AI in education is the ability to provide personalized learning experiences. AI tools can adapt to individual student needs, offering tailored explanations, examples, and practice problems based on the student's current understanding and progress. This personalized approach can help students:

\begin{itemize}
    \item \textbf{Identify Knowledge Gaps:} AI can help detect areas where students are struggling and provide targeted support to address these gaps, ensuring a more comprehensive understanding of the material.
    \item \textbf{Receive Immediate Feedback:} Through real-time analysis, AI tools can offer instant feedback on assignments and practice problems, allowing students to learn from their mistakes and improve more rapidly.
    \item \textbf{Customize Learning Paces:} Students can learn at their own pace with AI tools, allowing them to spend more time on challenging topics and move quickly through areas they find easier.
\end{itemize}

\subsection{Enhanced Understanding of Complex Concepts}

AI tools, particularly LLMs, are valuable for helping students grasp complex mathematical concepts by providing multiple explanations and perspectives. This can lead to a deeper understanding of the material. Specific benefits include:

\begin{itemize}
    \item \textbf{Multiple Explanatory Approaches:} AI can present different ways to approach a problem, catering to various learning styles and preferences. For example, it might offer a step-by-step solution, a conceptual explanation, or a visual representation.
    \item \textbf{Simplifying Difficult Topics:} AI tools can break down complex concepts into more manageable parts, making it easier for students to digest and understand challenging material.
    \item \textbf{Reinforcing Learning Through Repetition and Practice:} AI can generate endless practice problems and examples, allowing students to reinforce their understanding through repeated exposure and application.
\end{itemize}

\subsection{Engagement Through Interactive Learning}

AI and LLMs encourage active engagement with educational content, transforming passive learning into an interactive experience. This interactive element can significantly enhance student motivation and interest in the subject matter:

\begin{itemize}
    \item \textbf{Simulating Dialogues and Discussions:} AI tools can simulate conversations with students, encouraging them to ask questions and engage in discussions about mathematical theories, which can deepen their understanding and curiosity.
    \item \textbf{Encouraging Exploration and Experimentation:} With AI tools, students are free to explore different problem-solving methods and experiment with various strategies without the fear of making mistakes, fostering a growth mindset.
    \item \textbf{Gamification and Interactive Challenges:} Some AI tools incorporate gamified elements, such as quizzes and challenges, making learning more fun and engaging, which can enhance motivation and retention.
\end{itemize}

\subsection{Supporting Diverse Learning Needs}

AI technologies can support a diverse range of learning needs, making education more inclusive and accessible:

\begin{itemize}
    \item \textbf{Accessibility Features:} AI tools often come with features like text-to-speech, language translation, and alternative formats, which can assist students with disabilities or those learning in a second language.
    \item \textbf{Cultural and Linguistic Adaptation:} AI can provide culturally relevant examples and explanations, which can help students from different backgrounds relate to and understand the material better.
    \item \textbf{Providing Extra Help Outside of Class:} For students who need additional support outside of regular classroom hours, AI can act as a supplemental tutor, providing guidance and clarification at any time.
\end{itemize}

\subsection{Encouraging Collaborative Learning}

AI tools can facilitate collaborative learning by connecting students and fostering a community of learners:

\begin{itemize}
    \item \textbf{Facilitating Peer Collaboration:} AI platforms can connect students with peers for group study sessions, discussions, or collaborative problem-solving, enhancing learning through shared knowledge and teamwork.
    \item \textbf{Supporting Teachers in Collaborative Activities:} AI can help educators design and manage collaborative activities, such as group projects or peer reviews, by providing tools for monitoring progress and facilitating communication.
\end{itemize}

\subsection{Aiding in Research and Exploration}

AI tools can also be valuable for research purposes, helping students to explore advanced topics and engage in independent learning:

\begin{itemize}
    \item \textbf{Providing Access to Advanced Resources:} AI can guide students to relevant academic papers, articles, and resources, facilitating independent exploration and research.
    \item \textbf{Assisting with Data Analysis:} AI tools can help students analyze data, conduct experiments, and interpret results, making them valuable resources for research projects and assignments.
    \item \textbf{Fostering Curiosity and Lifelong Learning:} By providing easy access to a wide range of information, AI tools can encourage students to continue learning and exploring topics of interest beyond the classroom.
\end{itemize}

By leveraging these benefits, students can use AI tools to enhance their learning experiences, deepen their understanding of complex mathematical concepts, and develop essential skills for academic success.


\section{Proper Use of LLMs and AI in Academic Settings}

While AI tools and Large Language Models (LLMs) offer many benefits for enhancing learning, it is essential to use these technologies responsibly and ethically in an academic environment. This section provides guidelines for the appropriate use of AI and LLMs to ensure they serve as effective supplements to education without compromising academic integrity.

\subsection{Using AI as a Supplementary Tool}

AI should be seen as a supplementary tool to support and enhance learning, not as a replacement for engaging with course material or doing one's own work. The following guidelines can help students use AI tools effectively:

\begin{itemize}
    \item \textbf{Enhance Understanding, Not Replace It:} Use AI to clarify concepts, generate additional examples, or receive feedback, but always aim to understand the material independently. AI should help you think through problems, not think for you.
    \item \textbf{Balance AI Use with Traditional Learning Methods:} Continue to engage with textbooks, lectures, and other traditional resources. Use AI as one of many tools in your learning toolkit.
    \item \textbf{Set Learning Goals:} Clearly define what you want to achieve when using AI tools. For example, use AI to practice specific types of problems or to understand a particular concept more deeply, rather than using it for all your homework.
\end{itemize}

\subsection{Research and Understanding: Verifying Information}

AI tools can provide valuable insights and explanations, but it is crucial to verify the information they provide:

\begin{itemize}
    \item \textbf{Cross-Check with Reliable Sources:} Always compare AI-generated content with trusted academic resources, such as textbooks, peer-reviewed articles, or lecture notes, to ensure accuracy.
    \item \textbf{Understand the Source of Information:} Recognize that AI models generate responses based on patterns in the data they were trained on, which may include both reliable and unreliable sources. Use critical thinking to evaluate the quality and relevance of the information.
    \item \textbf{Seek Clarification and Context:} When using AI to explore new concepts, use it as a starting point for deeper investigation, rather than a definitive source. Follow up with your instructors or additional research if you have doubts or need more context.
\end{itemize}

\subsection{Avoiding Plagiarism and Maintaining Academic Integrity}

Maintaining academic integrity is essential in all aspects of your education. When using AI tools, it's important to ensure that all work submitted is your own:

\begin{itemize}
    \item \textbf{Do Not Copy and Paste AI-Generated Content:} Directly copying content produced by AI and submitting it as your own work constitutes plagiarism. Always rewrite information in your own words and ensure it reflects your understanding.
    \item \textbf{Acknowledge AI Assistance:} If you use AI to help generate ideas, solve problems, or provide explanations, you must disclose this in your work. Proper citation of AI tools is essential, especially if they significantly contributed to the content.
    \item \textbf{Understand Your Institution’s Policies:} Familiarize yourself with the academic integrity policies of your institution regarding the use of AI and LLMs. These policies outline the acceptable use of AI tools and the consequences of misuse.
\end{itemize}

\subsection{Transparency: Disclosing AI Use in Your Work}

Transparency about the use of AI tools is crucial for maintaining trust and integrity in an academic setting:

\begin{itemize}
    \item \textbf{Disclose AI Usage:} Clearly state when and how AI tools were used in your assignments or projects. For example, you might note that an AI was used to generate practice problems, provide initial explanations, or suggest additional resources.
    \item \textbf{Explain the AI’s Contribution:} Describe the role the AI played in your work. Did it help clarify a concept, provide a different perspective, or offer feedback on your approach? Be specific about how the AI influenced your learning and your final product.
    \item \textbf{Avoid Misrepresentation:} Do not imply that AI-generated work is entirely your own. Honesty about your use of AI tools fosters an ethical academic environment and promotes genuine learning.
\end{itemize}

\subsection{Ethical Considerations in Using AI}

Using AI tools responsibly also involves considering the ethical implications of their use:

\begin{itemize}
    \item \textbf{Be Aware of Biases:} AI tools may reflect biases present in their training data. Be mindful of this when using AI-generated content and critically evaluate its fairness and accuracy.
    \item \textbf{Respect Privacy and Security:} Avoid sharing personal, sensitive, or confidential information with AI tools, as they may not be secure or private.
    \item \textbf{Encourage Positive Use of AI:} Use AI tools to promote positive learning outcomes and avoid using them for unethical purposes, such as cheating or circumventing academic policies.
\end{itemize}

By following these guidelines, students can use AI tools in a way that enhances their education while upholding the values of honesty, integrity, and responsibility.


\section{Examples of Acceptable Use}

Using AI tools like ChatGPT can be beneficial for learning when done appropriately. Below are examples of acceptable use cases, along with sample conversations to guide students in how to effectively interact with AI while maintaining academic integrity.

\subsection{Brainstorming and Idea Generation}

\textbf{Example:} A student can use ChatGPT to brainstorm ideas for a math project on chaos theory.

\textbf{Conversation:}
\begin{mdframed}
\begin{flushleft}
\textbf{Student:} I need to come up with some ideas for a project on chaos theory in my math class. Can you help me brainstorm some topics?

\textbf{ChatGPT:} Sure! Here are a few ideas:
1. Explore the mathematical foundations of chaos theory, such as the butterfly effect and sensitive dependence on initial conditions.
2. Investigate real-world applications of chaos theory, like weather prediction, stock market analysis, or population dynamics.
3. Create a simulation that demonstrates chaotic behavior in a dynamical system, such as the Lorenz attractor.

\textbf{Student:} Thanks! I like the idea of exploring real-world applications. Can you help me narrow it down to one specific area?

\textbf{ChatGPT:} How about focusing on weather prediction? You could look into how small changes in initial conditions can lead to vastly different weather outcomes over time, illustrating the concept of chaos in meteorology.
\end{flushleft}
\end{mdframed}

\subsection{Clarifying Concepts Already Covered in Class}

\textbf{Example:} A student uses ChatGPT to better understand the concept of eigenvalues and eigenvectors, which was covered in class.

\textbf{Conversation:}
\begin{mdframed}
\begin{flushleft}
\textbf{Student:} I learned about eigenvalues and eigenvectors in my math class, but I'm still a bit confused. Can you explain what they are in simple terms?

\textbf{ChatGPT:} Absolutely! Eigenvalues and eigenvectors are concepts from linear algebra. In simple terms, an eigenvector is a vector that doesn't change its direction when a linear transformation is applied to it, although it might get stretched or shrunk. The eigenvalue is the factor by which the eigenvector is stretched or shrunk. If you think of a transformation as a function that moves and scales objects in space, eigenvectors point in the directions that remain unchanged by that transformation, while eigenvalues tell you how much the stretching or shrinking happens along those directions.

\textbf{Student:} That makes sense! Could you give an example of how this is used in a real-world application?

\textbf{ChatGPT:} Sure! One example is in facial recognition technology. Eigenvalues and eigenvectors are used in algorithms that recognize patterns in data, such as identifying faces in images. The technique, called Principal Component Analysis (PCA), uses eigenvectors to find the directions in which the data varies the most, which helps in reducing the dimensionality of the data and identifying the most significant features for recognizing faces.
\end{flushleft}
\end{mdframed}

\subsection{Practice Problems and Additional Explanations}

\textbf{Example:} A student practices solving integrals using ChatGPT for additional exercises and explanations.

\textbf{Conversation:}
\begin{mdframed}
\begin{flushleft}
\textbf{Student:} Can you provide me with a practice problem involving integration by parts?

\textbf{ChatGPT:} Certainly! Here's a practice problem for you:
Evaluate the integral \(\int x e^x \, dx\) using integration by parts.

\textbf{Student:} Let me try it. If I set \(u = x\) and \(dv = e^x dx\), then \(du = dx\) and \(v = e^x\). So, \(\int x e^x \, dx = x e^x - \int e^x \, dx = x e^x - e^x + C\).

\textbf{ChatGPT:} That's correct! Nice work. Integration by parts is a useful technique when you have a product of functions where one is easily differentiable (like \(x\)) and the other is easily integrable (like \(e^x\)).
\end{flushleft}
\end{mdframed}

\subsection{Simulating Discussions on Mathematical Theories}

\textbf{Example:} A student engages in a discussion with ChatGPT to understand the implications of G\"{o}del's Incompleteness Theorems.

\textbf{Conversation:}

\begin{mdframed}
\begin{flushleft}
\textbf{Student:} I've heard about G\"{o}del's Incompleteness Theorems, but I'm not sure what they imply for mathematics. Can you explain?

\textbf{ChatGPT:} G\"{o}del's Incompleteness Theorems are two fundamental results in mathematical logic that have profound implications for the foundations of mathematics. The first theorem states that in any consistent formal system that is capable of expressing basic arithmetic, there are statements that are true but cannot be proven within the system. The second theorem states that no such system can demonstrate its own consistency.

\textbf{Student:} Wow, that's interesting. Does this mean there are limitations to what we can know in mathematics?

\textbf{ChatGPT:} Exactly. G\"{o}del's theorems imply that there are always going to be some truths in mathematics that we cannot prove using a given set of axioms. This has a major impact on the philosophy of mathematics, suggesting that our mathematical knowledge is inherently incomplete and that there are always limits to what we can formally prove.
\end{flushleft}
\end{mdframed}




\section{Examples of Unacceptable Use}

While AI tools like ChatGPT can be valuable for learning, there are certain uses that violate academic integrity and are considered unacceptable. Below are examples of improper use cases, along with sample conversations to illustrate situations that should be avoided.

\subsection{Submitting AI-Generated Content as Original Work}

\textbf{Example:} A student uses ChatGPT to generate an entire assignment and submits it without modification or understanding.

\textbf{Conversation:}
\begin{mdframed}
\begin{flushleft}
\textbf{Student:} Can you write an essay on the history of calculus for me?

\textbf{ChatGPT:} Sure, here’s a brief essay:
Calculus is a branch of mathematics that deals with rates of change and the accumulation of quantities. Its origins can be traced back to ancient Greece, but it was developed into its modern form by Isaac Newton and Gottfried Wilhelm Leibniz in the 17th century...

\textbf{Student:} (Copies the text directly into their assignment without understanding or rephrasing any of the content.)
\end{flushleft}
\end{mdframed}

\textbf{Why This Is Unacceptable:} Submitting AI-generated content as your own work is plagiarism. It shows no engagement with the material and violates academic integrity policies.

\subsection{Relying Solely on AI Without Cross-Verification}

\textbf{Example:} A student blindly trusts an AI-generated solution without checking its correctness or understanding the underlying concepts.

\textbf{Conversation:}
\begin{mdframed}
\begin{flushleft}
\textbf{Student:} What is the derivative of \(e^{x^2}\)?

\textbf{ChatGPT:} The derivative of \(e^{x^2}\) is \(2xe^{x^2}\).

\textbf{Student:} (Writes down the answer without checking.) Okay, that’s my answer.

\textbf{Instructor’s Note:} The correct derivative of \(e^{x^2}\) is actually \(2xe^{x^2}\), but the student should always verify the process and not rely solely on AI answers.
\end{flushleft}
\end{mdframed}

\textbf{Why This Is Unacceptable:} Reliance on AI for answers without verification can lead to misunderstandings and the propagation of errors. It's important to cross-check AI responses with reliable sources and ensure comprehension of the material.

\subsection{Using AI to Bypass Learning or Understanding Key Concepts}

\textbf{Example:} A student uses ChatGPT to complete their homework without attempting to solve the problems themselves.

\textbf{Conversation:}
\begin{mdframed}
\begin{flushleft}
\textbf{Student:} What’s the solution to the equation \(2x + 3 = 7\)?

\textbf{ChatGPT:} To solve for \(x\), subtract 3 from both sides to get \(2x = 4\). Then divide both sides by 2 to find \(x = 2\).

\textbf{Student:} (Writes down the solution without understanding the steps involved or attempting to solve it independently.)
\end{flushleft}
\end{mdframed}

\textbf{Why This Is Unacceptable:} Using AI to bypass the learning process prevents the student from developing essential problem-solving skills. It undermines the purpose of education, which is to foster understanding and critical thinking.

\subsection{Misrepresenting AI-Assisted Work as Entirely Personal Effort}

\textbf{Example:} A student uses AI to generate ideas and explanations but does not disclose this assistance in their assignment.

\textbf{Conversation:}
\begin{mdframed}
\begin{flushleft}
\textbf{Student:} (Asks ChatGPT for a detailed explanation of Maxwell’s equations and uses the AI’s response as the main content for their assignment without citation.)

\textbf{ChatGPT:} Maxwell’s equations describe how electric and magnetic fields are generated by charges, currents, and changes of the fields. There are four equations...

\textbf{Student:} (Copies the content into their assignment and submits it, presenting it as their original work without mentioning the use of ChatGPT.)
\end{flushleft}
\end{mdframed}

\textbf{Why This Is Unacceptable:} Failing to acknowledge the use of AI tools in your work misrepresents the source of information and violates academic integrity. Transparency about the use of AI is crucial to maintain trust and honesty in an academic setting.

\subsection{Using AI for Unauthorized Assistance During Exams}

\textbf{Example:} A student uses ChatGPT to answer questions during a closed-book exam, which is against the rules.

\textbf{Conversation:}
\begin{mdframed}
\begin{flushleft}
\textbf{Student:} (During an exam) What is the integral of \(\sin(x)\) from 0 to \(\pi\)?

\textbf{ChatGPT:} The integral of \(\sin(x)\) from 0 to \(\pi\) is 2.

\textbf{Student:} (Uses this answer in their exam submission.)
\end{flushleft}
\end{mdframed}

\textbf{Why This Is Unacceptable:} Using AI tools during an exam violates exam rules and academic integrity policies. It constitutes cheating and can result in severe academic consequences.


\section{How to Cite AI Tools in Academic Work}

When using AI tools such as ChatGPT in academic work, it is important to properly acknowledge these tools to maintain transparency and academic integrity. This section provides guidelines on how to cite AI tools appropriately in different contexts.

\subsection{General Guidelines for Citing AI Tools}

Whenever you use an AI tool to assist with your work, you should:
\begin{itemize}
    \item Clearly state in your work that you have used an AI tool.
    \item Describe how the AI tool was used (e.g., for brainstorming, generating ideas, providing explanations, etc.).
    \item Provide a formal citation of the AI tool according to the appropriate citation style (e.g., APA, MLA, Chicago).
\end{itemize}

\subsection{Citation Formats for Different Styles}

Here are examples of how to cite ChatGPT in various citation styles:

\textbf{APA Style:}

\begin{mdframed}
\begin{flushleft}
OpenAI. (2024). \textit{ChatGPT} (GPT-4) [Large language model]. \url{https://chat.openai.com/}
\end{flushleft}
\end{mdframed}

\textbf{MLA Style:}

\begin{mdframed}
\begin{flushleft}
OpenAI. \textit{ChatGPT}. Version 4, OpenAI, 2024, \url{https://chat.openai.com/}.
\end{flushleft}
\end{mdframed}

\textbf{Chicago Style:}

\begin{mdframed}
\begin{flushleft}
OpenAI. 2024. \textit{ChatGPT} (GPT-4). Accessed September 4, 2024. \url{https://chat.openai.com/}.
\end{flushleft}
\end{mdframed}

\subsection{Examples of Proper Attribution in Academic Work}

\textbf{Example:} A student uses ChatGPT to help brainstorm ideas for an essay and wants to ensure proper attribution.

\textbf{How to Acknowledge AI Assistance:}

\begin{mdframed}
\begin{flushleft}
"In brainstorming for this essay, I used ChatGPT to generate initial ideas and outline the key points to cover. ChatGPT provided several topic suggestions, including an analysis of the socio-economic impacts of technological advancements."
\end{flushleft}
\end{mdframed}

\textbf{Example:} A student uses ChatGPT to clarify a concept in their assignment.

\textbf{How to Acknowledge AI Assistance:}

\begin{mdframed}
\begin{flushleft}
"ChatGPT was used to provide a clearer explanation of Maxwell’s equations. While the AI tool helped clarify some points, the final interpretation and analysis in this paper are my own."
\end{flushleft}
\end{mdframed}

\subsection{In-Text Citations and Footnotes}

For in-text citations or footnotes, mention the use of ChatGPT or any AI tool directly where it influenced your work:

\textbf{In-Text Citation Example:}

\begin{mdframed}
\begin{flushleft}
According to OpenAI’s ChatGPT, the derivative of \(e^{x^2}\) requires using the chain rule to find the correct solution (OpenAI, 2024).
\end{flushleft}
\end{mdframed}

\textbf{Footnote Example:}

\begin{mdframed}
\begin{flushleft}
1. Explanation provided by ChatGPT, OpenAI, Version 4, 2024, \url{https://chat.openai.com/}.
\end{flushleft}
\end{mdframed}

\subsection{Best Practices for AI Citations}

\begin{itemize}
    \item \textbf{Be Specific:} Describe exactly how the AI tool was used in your research or writing process.
    \item \textbf{Maintain Transparency:} Ensure that all uses of AI tools are clearly and accurately reported.
    \item \textbf{Follow Citation Guidelines:} Use the appropriate citation style as per your institution’s requirements.
\end{itemize}

By properly citing AI tools like ChatGPT, students uphold academic integrity, contribute to an ethical academic environment, and demonstrate respect for the sources of information and tools they use.


\section{Privacy and Ethical Considerations}

When using AI tools like ChatGPT, it is crucial to be mindful of privacy and ethical considerations. These tools can greatly assist in learning and research, but they also come with responsibilities related to data privacy and ethical use. This section provides guidelines on how to use AI tools responsibly, ensuring that personal information is protected and ethical standards are upheld.

\subsection{Protecting Personal Information}

AI tools often operate over the internet, which means that any information inputted can be stored, analyzed, or potentially exposed. Here are some guidelines to protect your privacy:

\begin{itemize}
    \item \textbf{Do Not Share Personal Information:} Avoid inputting personal details such as your full name, address, phone number, or any other sensitive information into AI tools.
    \item \textbf{Anonymize Data:} When discussing case studies or specific examples in academic work, anonymize any data that might relate to real people to protect their privacy.
    \item \textbf{Review Privacy Policies:} Before using any AI tool, review its privacy policy to understand how your data will be used and stored.
\end{itemize}

\textbf{Example:} A student inputs sensitive information into ChatGPT without considering privacy implications.

\textbf{Scenario:}
\begin{mdframed}
\begin{flushleft}
\textbf{Student:} My friend Jane Doe, who lives at 123 Main St, had a similar question. Can you help me explain how to solve it?

\textbf{ChatGPT:} While I can help with the question, it's important not to share personal details. To protect privacy, avoid inputting sensitive information.
\end{flushleft}
\end{mdframed}

\textbf{Why This Is Unacceptable:} Sharing personal information breaches privacy policies and can put yourself or others at risk. Always keep sensitive details private when using AI tools.

\subsection{Understanding Ethical Implications of AI Use}

AI tools can have biases based on the data they are trained on, and their use can have ethical implications:

\begin{itemize}
    \item \textbf{Be Aware of Biases:} AI models may reflect biases in their training data. Be critical of the information provided and cross-check facts with reliable sources.
    \item \textbf{Avoid Generating Harmful Content:} Do not use AI to generate or disseminate harmful, discriminatory, or offensive content.
    \item \textbf{Use AI Responsibly:} Leverage AI tools to enhance learning and research, but avoid using them in ways that could harm others or undermine ethical standards.
\end{itemize}

\textbf{Example:} A student tries to use ChatGPT to create harmful content.

\textbf{Scenario:}
\begin{mdframed}
\begin{flushleft}
\textbf{Student:} Can you help me write a convincing fake news article?

\textbf{ChatGPT:} I'm here to help with educational and constructive queries, but generating harmful or deceptive content isn't appropriate. Let's focus on ethical uses of AI.
\end{flushleft}
\end{mdframed}

\textbf{Why This Is Unacceptable:} Using AI to create harmful or misleading content violates ethical standards and can have serious consequences. AI should be used to promote positive and constructive outcomes.

\subsection{Respecting AI Tools' Limitations and Rules}

Understanding the limitations of AI tools and adhering to their usage guidelines is essential for ethical use:

\begin{itemize}
    \item \textbf{Follow Usage Guidelines:} Each AI tool has specific guidelines and terms of service. Make sure you understand and follow them.
    \item \textbf{Acknowledge AI’s Limitations:} AI tools are not perfect and can make mistakes. Always review and verify the information provided.
    \item \textbf{Encourage Positive and Ethical Use:} Use AI to support ethical goals, such as learning, research, and constructive dialogue.
\end{itemize}

\textbf{Example:} A student attempts to use ChatGPT in a way that violates its terms of service.

\textbf{Scenario:}
\begin{mdframed}
\begin{flushleft}
\textbf{Student:} Can you help me hack into a website?

\textbf{ChatGPT:} I cannot assist with that request. It's important to use AI tools ethically and within the bounds of the law.
\end{flushleft}
\end{mdframed}

\textbf{Why This Is Unacceptable:} Misusing AI tools to break the law or violate ethical guidelines is not permitted. It's crucial to understand and respect the limitations and rules associated with AI tools.

\subsection{Best Practices for Ethical AI Use}

\begin{itemize}
    \item \textbf{Think Critically:} Always critically evaluate the output from AI tools, especially when it involves sensitive or controversial topics.
    \item \textbf{Stay Informed:} Keep up-to-date with developments in AI ethics and privacy to use these tools responsibly.
    \item \textbf{Promote Positive Use:} Encourage others to use AI tools ethically and for constructive purposes.
\end{itemize}

By adhering to these guidelines, students can ensure that their use of AI tools is both ethical and respectful of privacy, contributing to a positive academic environment.


\section{Conclusion}

The integration of Large Language Models (LLMs) and artificial intelligence (AI) into educational settings has the potential to significantly enhance the learning experience. By providing personalized support, enabling deeper understanding through diverse explanatory methods, and fostering interactive and engaging learning environments, these tools can revolutionize how students connect to knowledge.

Throughout this document, we have explored both the opportunities and the responsibilities that come with using AI in an academic context. Proper use of AI tools involves understanding their capabilities and limitations, ensuring academic integrity by avoiding plagiarism, citing AI-generated content appropriately, and considering privacy and ethical implications.

\subsection{The Positive Impact of AI on Education}

When used responsibly, AI and LLMs can:

\begin{itemize}
    \item \textbf{Enhance Learning:} AI can provide immediate feedback, tailor learning experiences to individual needs, and offer a variety of perspectives on complex topics, enriching the educational process.
    \item \textbf{Promote Accessibility:} AI tools can make education more accessible by offering support in multiple languages, providing accommodations for diverse learning needs, and making high-quality educational resources widely available.
    \item \textbf{Encourage Exploration and Curiosity:} By facilitating easy access to a vast range of information and fostering a spirit of inquiry, AI tools can inspire students to explore new subjects and deepen their understanding.
\end{itemize}

\subsection{The Transformative Potential of AI and LLMs}

AI and LLMs hold the potential to revolutionize education by transforming how we connect to knowledge. These technologies enable a more dynamic and interactive approach to learning, where students are not just passive recipients of information but active participants in their educational journey. The ability to engage with AI in a conversational format allows for a more personalized and responsive learning experience, helping students to learn more effectively and efficiently.

As AI continues to evolve, its role in education is likely to expand, offering even more innovative ways to support teaching and learning. However, it is essential that we approach this evolution thoughtfully, with a commitment to ethical use and a focus on enhancing the educational experience for all.

\subsection{Encouragement for Responsible Use}

We encourage all students and educators to embrace the possibilities that AI and LLMs present while remaining mindful of their responsibilities. By using these tools ethically, transparently, and with an awareness of privacy and ethical considerations, we can create a positive learning environment that benefits everyone.

AI is a powerful tool that, when used correctly, can greatly enhance our educational experiences and expand our understanding of the world. As we continue to integrate these technologies into our academic practices, let us strive to do so in ways that uphold the highest standards of integrity and respect for knowledge.

\subsection{Looking Ahead}

The future of education is bright with the integration of AI and LLMs, offering new ways to connect with knowledge and each other. By harnessing the potential of these tools, we can foster a more inclusive, engaging, and effective educational experience for all learners.

In conclusion, AI tools like ChatGPT provide a valuable resource for enhancing learning and exploration. However, their use must be guided by principles of honesty, integrity, and ethical responsibility. Together, we can navigate this new landscape and ensure that AI serves as a positive force in education, driving innovation and deeper connections to knowledge.

\begin{flushright}
\textit{Community College Mathematics Department}
\end{flushright}



\section{Additional Resources}
For further reading on AI ethics and academic integrity, consider the following resources:
\begin{itemize}
    \item \href{https://en.wikipedia.org/wiki/Ethics_of_artificial_intelligence}{Ethics of Artificial Intelligence}
    \item \href{https://plato.stanford.edu/entries/ethics-ai/}{Stanford Encyclopedia of Philosophy: Ethics of AI}
\end{itemize}

\section*{Acknowledgments}

This document was created with the assistance of ChatGPT, a large language model developed by OpenAI, to generate content and provide ideas. Proper attribution is given in accordance with our academic guidelines on the use of AI tools.

\textbf{Citation:}

OpenAI. (2024). \textit{ChatGPT} (GPT-4) [Large language model]. \url{https://chat.openai.com/}


\end{document}
